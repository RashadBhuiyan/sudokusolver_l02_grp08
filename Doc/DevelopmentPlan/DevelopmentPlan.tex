\documentclass[11pt]{article}
\usepackage[utf8]{inputenc}
\usepackage[letterpaper, total={6in, 9in}]{geometry}
\usepackage[table,xcdraw]{xcolor}
\usepackage{hyperref}

\usepackage{booktabs}
\usepackage{tabularx}
\usepackage[normalem]{ulem}

\title{SE 3XA3: Development Plan\\Sudoku Solver}

\author{Team 08, SudoCrew
		\\ Rashad Bhuiyan (bhuiyr2)
		\\ Stanley Chan (chans67)
		\\ Kai Zhu (zhuk2)
}

\date{}

%\input{../Comments}

\begin{document}

\maketitle

 As avid Sudoku lovers and software engineers, we seek to implement an accessible platform capable of providing and validating solutions using minimal manual input such as an uploaded photograph of a puzzle. This document highlights all necessary information for the control of the project. Furthermore, this document states guidelines and roles for each member so that they understand their due dates and deliverable.

\section{Team Meeting Plan}
All team members are expected to meet twice a week during the specified 3XA3 lab sessions on Monday and Thursday for 2 hours, totalling 4 hours a week. These meetings will be held either in-person at the Information Technology Building in room 236 or virtually through Discord, given the circumstances that week. Any additional work required will be done individually outside of meeting times. During these meetings, discussions regarding project design, implementation, as well as any necessary documentation will occur. Rashad will be responsible for chairing the meetings while Kai and Stanley will be responsible for updating the agenda and taking meeting minutes. \textcolor{red}{Progress will be tracked through the use of Gantt charts, which will be updated accordingly depending on which milestones and targets are met during each meeting} Should a member be unable to attend a meeting, they must notify all other members as soon as possible. 

\section{Team Communication Plan}
Communication will be done through a combination of Discord and Microsoft Teams. Each member is expected to respond in a timely manner. Microsoft Teams will be used for formal communication with TAs and the professor while in-team communication will be done through Discord. Git issues will also be used to keep track of current workflow within the project and staying updated on what needs to be done.

\section{Team Member Roles}
Rashad will be responsible for GitLab organization and chairing the meetings. Kai will be responsible for resolving Git conflicts and updating the agenda. Stanley will be in charge of meeting minutes and initial documentation. All members will be responsible for thorough documentation and development.

\begin{table}[h]
\centering
\begin{tabular}{|
>{\columncolor[HTML]{FFFFFF}}l |
>{\columncolor[HTML]{FFFFFF}}l |}
\hline
\multicolumn{1}{|c|}{\cellcolor[HTML]{FFFFFF}\textbf{Technology}} & \multicolumn{1}{c|}{\cellcolor[HTML]{FFFFFF}\textbf{Expert}} \\ \hline
Documentation                                       & Stanley                                              \\ \hline
\multicolumn{1}{|c|}{\cellcolor[HTML]{FFFFFF}Git}   & \multicolumn{1}{c|}{\cellcolor[HTML]{FFFFFF}Rashad}  \\ \hline
\multicolumn{1}{|c|}{\cellcolor[HTML]{FFFFFF}LaTeX} & \multicolumn{1}{c|}{\cellcolor[HTML]{FFFFFF}Stanley} \\ \hline
Flask                                               & Rashad                                               \\ \hline
OpenCV                                              & Kai                                                  \\ \hline
JavaScript                                          & Kai                                                  \\ \hline
\end{tabular}
\end{table}

\section{Git Workflow Plan}
The team will be using GitLab for version control. \sout{Every team member will work on their own separate branches} \textcolor{red}{A new branch will be dedicated to every additional feature of the program, separate from the main branch}. Once testing is done and there are no errors, it will be merged to the main branch when it is approved by a pull request. Once the repository is ready to be marked for a deliverable, it will be tagged using the appropriate convention mentioned in class and a final commit will be done in order to push for submission.

\section{Proof of Concept Demonstration Plan}

Implementing an effective Sudoku board interpreter using OpenCV will be the biggest hurdle of this project, as much of the interpretation is dependent on how the Sudoku boards being solved are drawn. For example, the differences in Sudoku boards found in newspapers from Sudoku boards found in dedicated Sudoku books could result in the program only recognizing the Sudoku board from one medium and not the other. In order to make the application as user friendly as possible, the solver must be able to interpret Sudoku boards from varying mediums. Our group is experienced in regards to the other technologies being used in this project, so implementation in those areas should not be as big of a challenge. \\

In addition to using Pytest as the main test framework, we will also be using sample Sudoku boards to test both the solver's functionality as well as the application's ability to interpret Sudoku boards.

\section{Technology}

Sudoku Solver will be written and developed using Python, with the front-end portion of the application being developed with HTML, CSS, and JavaScript, and Python Flask as the web framework. The project will also use OpenCV to allow for interpretation of Sudoku boards from uploaded pictures. To test the \textcolor{red}{backend portion of the }program, we will be using the Pytest framework. \sout{Pydoc will be used for documentation of code and document generation.}

\section{Coding Style}

Sudoku Solver will be using the PEP-8 coding style. Information for the coding style can be found at https://www.python.org/dev/peps/pep-0008/. 

\section{Project Schedule}

Here is the link to our  \href{https://gitlab.cas.mcmaster.ca/bhuiyr2/sudokusolver_l02_grp08/-/blob/main/ProjectSchedule/Gantt_Sudoku.gan}{Gantt Chart}.

\section{Project Review}
Functional and non-functional requirements listed in the SRS were fully fulfilled and tested using a traceability matrix. We accomplished all major goals of of building a comprehensive Sudoku tool set featuring computer-vision board recognition, solution validation, game generation, while also delivering quality-of-life features such as the multi-platform, responsive design, an intuitive and stylish user interface, and fast puzzle-solving and generation implementations. \\

Obstacles that we encountered along the way include new technologies previously unfamiliar to us, such as computer vision, machine learning, responsive frontend web application design. We overcame these difficulties through utilizing specialties of each team member, and a rapid process of learning, building, and testing. The time constraints were managed by breaking the projects into smaller definitive deadlines using a Gantt chart.
\\

The project proceeded mostly as planned except with the change of dropping the React frontend design, which in retrospect appears superfluous such that removing it did not impact the quality or the functionalities of our project. \\

An area that may be improved in future development is a greater use of automatic testing tools. The nature of computer vision made automatic testing difficult for many components of the project, but exploring new tools and methods to implement more automatic testing for the rest of the application would speed up the development and improve robustness.

\section{Revision History}
\begin{table}[hp]
\caption{Revision History} \label{TblRevisionHistory}
\begin{tabularx}{\textwidth}{llX}
\toprule
\textbf{Date} & \textbf{Developer(s)} & \textbf{Change}\\
\midrule
2022-02-03 & All & Initial commit for draft\\
2022-04-09 & All & Updated Development Plan; new text in red, deprecated text struck out\\
2022-04-012 & All & Added Project Review\\
\bottomrule
\end{tabularx}
\end{table}

\end{document}