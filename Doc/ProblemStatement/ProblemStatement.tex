\documentclass[11pt]{article}
\usepackage[utf8]{inputenc}
\usepackage[letterpaper, total={6in, 9in}]{geometry}

\title{Sudoku Solver: Problem Statement}
\author{Rashad Bhuiyan (400263180), Kai Zhu , Stanley Chan (400257827)}
\date{January 28, 2022}

\begin{document}

\maketitle

\section{Introduction}
Sudoku is a popular logic-based game most often played on physical media such as books and newspapers. Aside from its entertainment value, Sudoku is also a problem frequently of interest in computer science, featured in over 33,000 academic articles on Google Scholar. An inconvenience that players face is a lack of timely access to published solutions, resulting in the absence of immediate feedback. Additionally, the process of manually checking for solution correctness is tedious and prone to error. As avid Sudoku lovers and software engineers, we seek to implement an accessible platform capable of providing and validating solutions using minimal manual input such as an uploaded photograph of a puzzle.


\section{Importance}
Existing Sudoku solvers either involve creating their own sets of Sudoku boards randomly, or have a static limit in terms of the number of boards available. Furthermore, inputting incomplete Sudoku boards manually on an existing Sudoku solver is tedious and prone to error. We aim to solve by giving users the ability to upload Sudoku boards through the use of an image, allowing for more flexibility by giving the users an option to choose a specific Sudoku board that they would want solved. 
\\
\\
All of these features are to be delivered to users on a sleek and clean user interface, allowing for ease of input and can subsequently be used by a wider audience. Having a nice user interface allows for users to navigate the options and understand the results easily, ensuring that our project is properly used and understood.

\section{Context}
With the implementation of an easy to use user interface, Sudoku Solver is to be used by audiences of all ages, in particular, those who have an interest in solving Sudoku boards. Given both its uses in entertainment and education, stakeholders may come in the form of everyday Sudoku fans, computer scientists, and algorithm enthusiasts. Our group intends to implement the program as a web application, allowing end users to access the solver through a variety of platforms that support the use of modern day browsers, which include Windows, Linux, MacOS, and mobile devices.

\end{document}