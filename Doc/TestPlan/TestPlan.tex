\documentclass[11pt]{article}
\usepackage[utf8]{inputenc}
\usepackage[letterpaper, total={6in, 9in}]{geometry}
\usepackage[table,xcdraw]{xcolor}
\usepackage{hyperref}
\usepackage{adjustbox,lipsum}

\usepackage{booktabs}
\usepackage{float}
\usepackage{tabularx}
\usepackage{hyperref}
\hypersetup{
    colorlinks,
    citecolor=black,
    filecolor=black,
    linkcolor=red,
    urlcolor=blue
}
\usepackage[round]{natbib}

\graphicspath{ {Assets/} }

\title{SE 3XA3: Test Plan\\Sudoku Solver}

\author{Team 08, SudoCrew
		\\ Rashad A. Bhuiyan (bhuiyr2)
		\\ Kai Zhu (zhuk2)
		\\ Stanley Chan (chans67)
}

\date{\today}

% \input{../Comments}

\begin{document}

\maketitle

\pagenumbering{roman}
\tableofcontents
\listoftables
\listoffigures

\newpage

\begin{table}[H]
\caption{\bf Revision History}
\begin{tabularx}{\textwidth}{p{3cm}p{2cm}X}
\toprule {\bf Date} & {\bf Version} & {\bf Notes}\\
\midrule
2022-02-28 & 0.0 & Created Test Plan; started on General Information\\
2022-03-09 & 0.1 & Updated Plan and Appendix, started System Test Description\\
2022-03-11 & 0.2 & Updated Sections 3-6\\
\bottomrule
\end{tabularx}
\end{table}

\newpage

\pagenumbering{arabic}

\section{General Information}

\subsection{Purpose}
The purpose of this document is to describe the testing, validation, and verification procedures that will be implemented for our Sudoku Solver program. The majority of the unit testing will be accomplished through PyTest as our program is primarily written using Python. Structural testing will be done using individually-created testing classes for proper analysis of branching. Non-functional requirements will primarily be realized through manual testing as there is a heavy emphasis on visualization of our project. System testing is done before proof-of-concept demonstration, revision 0 demonstration, and the final demonstration.

\subsection{Scope}
This test plan provides a basis for testing the functionality of the extended implementation and every new addition to the original Sudoku Solver algorithm by TechWithTim. Since the extended implementation involves creating a web application and methods of playing the game, the scope of testing covers web-application navigation and management, Sudoku generation and solution algorithms, and image recognition algorithms through machine learning. This document provides testing methodologies that covers the scope of the program as well as outlining all methods and tools used for testing.

\subsection{Acronyms, Abbreviations, and Symbols}
\begin{table}[H]
\caption{\bf Table of Naming Conventions and Terminology} \label{Table}
\centering
\begin{tabularx}{\textwidth}{p{3cm}X}
\toprule
Terminology     & Meaning \\
\midrule
PoC & Proof of Concept, one of the deliverables of the 3XA3 project\\
SRS & Software Requirements Specification, the document outlining all requirements for this program\\
GUI & graphical user interface, the front-end of the program that users can view and interact\\
Structural Test & Testing that focusses on the internal structure and branching of the software\\
Functional Test & Testing derived from the SRS\\
Dynamic Test & Testing that involves test cases running during execution\\
Static Test & None-code testing done through code inspection\\
Manual Test & Testing that is done manually by people\\
Automated Test & Testing that is done using testing software such as PyTest\\
Unit Test & Testing that focusses on individual functions and methods\\
System Test & Testing the entire system as a whole rather than individual components\\
UI     & User interface, the interface that allows for user interaction with the system. \\
PyTest & A Python library designed to help create tests for Python code. \\
MNIST & A Modified National Institute of Standards and Technology data set with a large number of images of handwritten digits.
\bottomrule
\end{tabularx}
\end{table}

\subsection{Overview of Document}
This document contains all information regarding the testing plan for the Sudoku Solver project. An overall plan will be addressed through a schedule as well as creation of a testing team. Furthermore, every functional and non-functional requirement from the SRS will be addressed and tested using various form of blackbox and whitebox testing. PoC testing will cover a specific sample of methods and will highlight some key differences of implementation between the original project and the current implementation.

\newpage

\section{Plan}
	
\subsection{Software Description}
The purpose of this software application is to provide a comprehensive suite of tools for generating, recognizing, and solving Sudoku puzzles. The application will provide a web-based front-end with an intuitive interface to cater to users of different technical abilities on most modern hardware. Computer vision is also utilized to improve ease of use, by directly interfacing puzzles from print-media to the application.

\subsection{Test Team}
The team responsible for testing consists of Rashad Bhuiyan, Stanley Chan, and Kai Zhu.

\subsection{Automated Testing Approach}

\subsection{Testing Tools}

\subsection{Testing Schedule}
		
Detailed testing schedule and responsibilities are included in the Project Gantt chart, available at  \url{https://gitlab.cas.mcmaster.ca/bhuiyr2/sudokusolver_l02_grp08/-/raw/main/ProjectSchedule/Gantt_Sudoku.pdf}

\newpage

\section{System Test Description}
	
\subsection{Tests for Functional Requirements}

\subsubsection{Homepage Navigation}
		
\paragraph{Display Buttons}

\begin{enumerate}

\item{FS-DB-1\\}

Type: Functional, Dynamic, Manual
					
Initial State: Looking at homepage of Flask app
					
Input: Click Manual or Image Input button
					
Output: Appropriate screen will be shown depending on input type specified
					
How test will be performed: The button that specifies either 'Manual Input' or 'Image Input' will be pressed to see if it will redirect the user to the appropriate page that allows them to either upload an image or manually input a Sudoku board.
					
\item{FS-DB-2\\}

Type: Functional, Dynamic, Manual
					
Initial State: Looking at homepage of Flask app
					
Input: Click play a game button
					
Output: Sudoku game for user to play will be loaded
					
How test will be performed: The button that says 'Play a Game' will be pressed to check and see if it will redirect the user to a separate page where they may play a game of Sudoku.

\item{FS-DB-3\\}

Type: Functional, Dynamic, Manual
					
Initial State: Old browser
					
Input: Enter Sudoku Solver website path
					
Output: Error message stating browser mismatch
					
How test will be performed: The website will attempt to be launched using an old and/or deprecated browser, such as Internet Explorer, where an error message will be shown stating a mismatch of browser and website.

\end{enumerate}

\subsubsection{Image Upload}

\paragraph{Initial Upload}

\begin{enumerate}

\item{FS-IU-1\\}

Type: Functional, Dynamic, Manual
					
Initial State: Sudoku Image input page with image uploaded
					
Input: Submit image
					
Output: Loading symbol will be shown
					
How test will be performed: The change in website state will be tested by uploading and submitting a Sudoku board to see if a loading icon is displayed.
					
\item{FS-IU-2\\}

Type: Functional, Dynamic, Manual
					
Initial State: Sudoku Image input page with image uploaded
					
Input: Submit Sudoku board
					
Output: Converted image to array
					
How test will be performed: The functions used to interpret the image will be tested by giving an image of Sudoku board and checking output flags to see if the image has been able to be converted to an array.

\end{enumerate}

\paragraph{Image Interpretation}

\begin{enumerate}

\item{FS-II-1\\}

Type: Functional, Dynamic, Manual
					
Initial State: Loading icon after image upload
					
Input: Invalid image was originally given
					
Output: Error message indicating that image could not be interpreted
					
How test will be performed: An invalid image will be uploaded to check whether the function can successfully recognize that it is invalid and output an error message to the user that the image is invalid or could not be recognized. This checks code coverage of the invalid branching path for the image recognition algorithm.
					
\item{FS-II-2\\}

Type: Functional, Dynamic, Manual
					
Initial State: Loading icon after image upload
					
Input: Valid image was given
					
Output: Array of Sudoku numbers to solver algorithm
					
How test will be performed: The function to interpret a Sudoku board will be given a valid Sudoku board to check whether it will send the correct integer array to the Sudoku solver algorithm. This checks code coverage of the valid branching path for the image recognition algorithm.

\end{enumerate}

\paragraph{Solution Output}

\begin{enumerate}

\item{FS-SO-1\\}

Type: Functional, Dynamic, Manual
					
Initial State: Loading icon after image submission
					
Input: Legal image with a possible solution
					
Output: Display solution to user
					
How test will be performed: The overall upload system will be tested using a valid Sudoku board with a known solution where the generated Sudoku solution will be compared to the actual solution to determine whether the interpretation and solution algorithms work effectively. This checks code coverage of the valid branching path for the Sudoku solution algorithm.
					
\item{FS-SO-2\\}

Type: Functional, Dynamic, Manual
					
Initial State: Loading icon after image submission
					
Input: Legal image with no solution
					
Output: Notify user about no solution
					
How test will be performed: The overall upload system will be tested using an invalid Sudoku board with no known solution to check whether the image interpretation algorithm and the solution algorithm will notify the user that there is no solution possible. This checks code coverage of the invalid branching path for the Sudoku solution algorithm.

\end{enumerate}

\subsubsection{Manual Input}

\paragraph{Board Display}

\begin{enumerate}

\item{FS-BD-1\\}

Type: Functional, Dynamic, Manual
					
Initial State: Homepage of Sudoku Solver Flask app
					
Input: Click on Manual Input button
					
Output: Empty Sudoku board on Manual Input page
					
How test will be performed: The navigation to the Manual Input page will be tested alongside whether an empty grid is shown for the user to input their numbers of the Sudoku board.
					
\item{FS-BD-2\\}

Type: Functional, Dynamic, Manual
					
Initial State: Manual Input page with empty Sudoku board
					
Input: Number in cell
					
Output: Number appears in cell
					
How test will be performed: The manual input function will be tested by attempting to place the numbers 1-9 in any empty cell of the Sudoku board.

\end{enumerate}

\paragraph{Solution Verification}

\begin{enumerate}

\item{FS-SV-1\\}

Type: Functional, Dynamic, Manual
					
Initial State: Semi-populated Sudoku board
					
Input: Number that breaks Sudoku rules
					
Output: Notification about invalid input
					
How test will be performed: The manual input will be tested to see whether it can reject bad input by putting in an invalid input into a row/column/box due to a repeating number and checking if the function notifies the user about an invalid move. This checks code coverage of the invalid input branching path for the manual input function.
					
\item{FS-SV-2\\}

Type: Functional, Dynamic, Manual
					
Initial State: Semi-populated board
					
Input: Submit button
					
Output: Loading icon
					
How test will be performed: The submission function will be tested by filling out the manual input with a legal board and pressing the submit button to see if a loading icon will appear to indicate that the given board is being processed.

\item{FS-SV-3\\}

Type: Functional, Dynamic, Manual
					
Initial State: Loading icon after manual submission
					
Input: Legal board with no solution
					
Output: Notify user about no solution
					
How test will be performed: The overall upload system will be tested using an invalid Sudoku board with no known solution to check whether the solution algorithm will notify the user that there is no solution possible. This checks code coverage of the invalid branching path for the Sudoku solution algorithm.

\item{FS-SV-4\\}

Type: Functional, Dynamic, Manual
					
Initial State: Loading icon after manual submission
					
Input: Legal board with possible solution
					
Output: Display solution to user
					
How test will be performed: The overall upload system will be tested using a valid Sudoku board with a known solution to check whether the solution algorithm will produce the same solution as the previously known solution. This checks code coverage of the valid branching path for the Sudoku solution algorithm.

\end{enumerate}

\subsubsection{Play Sudoku}

\paragraph{Board Generation}

\begin{enumerate}

\item{FS-BG-1\\}

Type: Functional, Dynamic, Manual
					
Initial State: Sudoku Gameplay page
					
Input: Click button to start game
					
Output: Loading symbol shown
					
How test will be performed: Ensure that when the user presses the button, the website displays a loading symbol.
					
\item{FS-BG-2\\}

Type: Functional, Dynamic, Automated
					
Initial State: Board is valid
					
Input: Completed valid Sudoku board
					
Output: true boolean value, indicating that the board is valid
					
How test will be performed: A completed valid Sudoku board will be used as an input into the function. The test will ensure that the output of the function will be true.

\item{FS-BG-3\\}

Type: Functional, Dynamic, Automated
					
Initial State: Board is valid
					
Input: Completed invalid Sudoku board
					
Output: false boolean value, indicating that the board is invalid
					
How test will be performed: A completed Sudoku board that does not follow the Sudoku rules will be used as an input into the function. The test will ensure that the output of the function will be false.

\item{FS-BG-4\\}

Type: Functional, Dynamic, Automated
					
Initial State: Board is valid
					
Input: Incomplete valid/invalid Sudoku board
					
Output: an exception should be raised, indicating that the board is incomplete
					
How test will be performed: An incomplete Sudoku board will be used as an input into the function. The test will ensure that the function will raise an exception.

\item{FS-BG-5\\}

Type: Functional, Dynamic, Automated
					
Initial State: Board has no solutions
					
Input: No solution board
					
Output: false boolean value, indicating that the board has no solutions
					
How test will be performed: A Sudoku board with no solutions will be used as input for the function. The test will ensure that the function outputs false.

\item{FS-BG-6\\}

Type: Functional, Dynamic, Automated
					
Initial State: Board has no solutions
					
Input: Board with solution(s)
					
Output: true boolean value, indicating that the board has solutions
					
How test will be performed: A Sudoku board with one or more solutions will be used as input for the function. The test will ensure that the function outputs true.

\item{FS-BG-7\\}

Type: Functional, Dynamic, Automated
					
Initial State: Board is unique
					
Input: Non-unique board (multiple solutions)
					
Output: false boolean value, indicating that the board is not unique
					
How test will be performed: A Sudoku board with more than one solution will be used as input for the function. The test will ensure that the function outputs false.

\item{FS-BG-8\\}

Type: Functional, Dynamic, Automated
					
Initial State: Board is unique
					
Input: Unique board (one solution)
					
Output: true boolean value, indicating that the board is unique
					
How test will be performed: A Sudoku board with only one solution will be used as input for the function. The test will ensure that the function outputs true.

\item{FS-BG-9\\}

Type: Functional, Dynamic, Automated
					
Initial State: Board is unique
					
Input: No solution board
					
Output: exception should be raised, informing the user that the board has no solutions
					
How test will be performed: A Sudoku board with no solutions will be used as input for the function. The test will ensure that the function raises an exception, and informs the user that the Sudoku board has no solutions.

\item{FS-BG-10\\}

Type: Functional, Dynamic, Automated
					
Initial State: No board generated
					
Input: Number of hints
					
Output: A randomly generated unique Sudoku board
					
How test will be performed: Generate multiple Sudoku boards with the function, and test that each board generated has a solution and is unique (with the aforementioned functions). Also iterate through each board and ensure that the number of hints generated matches the input. 

\end{enumerate}

\paragraph{Sudoku Gameplay}

\begin{enumerate}

\item{FS-SG-1\\}

Type: Functional, Dynamic, Manual, Static etc.
					
Initial State: 
					
Input: 
					
Output: 
					
How test will be performed: 
					
\item{FS-SG-2\\}

Type: Functional, Dynamic, Manual, Static etc.
					
Initial State: 
					
Input: 
					
Output: 
					
How test will be performed: 

\item{FS-SG-3\\}

Type: Functional, Dynamic, Manual, Static etc.
					
Initial State: 
					
Input: 
					
Output: 
					
How test will be performed: 

\end{enumerate}

\paragraph{User Sudoku Verification}

\begin{enumerate}

\item{FS-USV-1\\}

Type: Functional, Dynamic, Manual, Static etc.
					
Initial State: 
					
Input: 
					
Output: 
					
How test will be performed: 
					
\item{FS-USV-2\\}

Type: Functional, Dynamic, Manual, Static etc.
					
Initial State: 
					
Input: 
					
Output: 
					
How test will be performed: 

\end{enumerate}

\subsection{Tests for Nonfunctional Requirements}

\subsubsection{Look and Feel}
		
\paragraph{Homepage Display}

\begin{enumerate}

\item{SS-HD-1\\}

Type: Non-Functional, Static, Manual
					
Initial State: Homepage of the Sudoku solver Flask app
					
Input: Users from testing group will view the homepage
					
Output: Users will give feedback on button layout
					
How test will be performed: The users from the testing group will each load the Homepage on their own device to check the buttons and see if they look distinct and separate from each other.
					
\item{SS-HD-2\\}

Type: Non-Functional, Static, Manual
					
Initial State: Homepage of the Sudoku solver Flask app
					
Input: Users from the test group will change the browser size
					
Output: The Homepage scales accordingly, making UI changes based on the dimensions of the window
					
How test will be performed: The users from the testing group will each load the Homepage on their own device and change the browser's size to see whether the page scales accordingly and changes views depending on screen dimensions.

\end{enumerate}

\paragraph{User Interface Design}

\begin{enumerate}

\item{SS-UID-1\\}

Type: Non-Functional, Static, Manual
					
Initial State: Homepage of the Sudoku solver Flask app
					
Input: Users from the test group view the web app
					
Output: Users give feedback on the overall UI
					
How test will be performed: The users from the testing group will each load the Homepage on their own device and explore the web app on their own, taking note of each UI element and whether it is appropriate.
					
\item{SS-UID-2\\}

Type: Non-Functional, Static, Manual
					
Initial State: Homepage of the Sudoku solver Flask app
					
Input: Users with mild forms of colour-blindness view the web app
					
Output: Users give feedback on the accessibility of the web app
					
How test will be performed: The users officially diagnosed with some forms of colour-blindness will load the Homepage on their own device and explore the web app on their own, taking note of whether every page is accessible and appealing to them.

\end{enumerate}

\subsubsection{Usability and Humanity}

\paragraph{Ease of Use}

\begin{enumerate}

\item{SS-EU-1\\}

Type: Non-Functional, Static, Manual
					
Initial State: Manual input page or Image Upload page
					
Input: Users from test group follow instructions of respective page
					
Output: System accepts provided Sudoku board
					
How test will be performed: The users from the testing group will check both the Manual Input and the Image Upload pages by following the given instructions about filling out the Sudoku board and submitting them in order to get a valid solution, using a set of Sudoku boards that are both valid and invalid.
					
\item{SS-EU-2\\}

Type: Non-Functional, Static, Manual
					
Initial State: Sudoku gameplay page
					
Input: Users from test group start playing the game
					
Output: Users play Sudoku and are given solution at the end
					
How test will be performed: The users from the testing group will attempt to play a game of Sudoku on the Sudoku gameplay page and check whether their given solution is correct and whether the game then provides the actual solution.

\end{enumerate}

\paragraph{Ease of Learning}

\begin{enumerate}

\item{SS-EL-1\\}

Type: Non-Functional, Static, Manual
					
Initial State: How to Play Sudoku page
					
Input: Users from the test group read the page 
					
Output/Result: 
					
How test will be performed: 
					
\item{SS-EL-2\\}

Type: Non-Functional, Static, Manual
					
Initial State: 
					
Input: 
					
Output: 
					
How test will be performed: 

\end{enumerate}

\subsubsection{Performance}

\paragraph{Speed}

\begin{enumerate}

\item{SS-S-1\\}

Type: Non-Functional, Static, Manual
					
Initial State: 
					
Input/Condition: 
					
Output/Result: 
					
How test will be performed: 
					
\item{SS-S-2\\}

Type: Non-Functional, Static, Manual
					
Initial State: 
					
Input: 
					
Output: 
					
How test will be performed: 

\end{enumerate}

\paragraph{Precision}

\begin{enumerate}

\item{SS-P-1\\}

Type: Non-Functional, Static, Manual
					
Initial State: 
					
Input/Condition: 
					
Output/Result: 
					
How test will be performed: 
					
\item{SS-P-2\\}

Type: Non-Functional, Static, Manual
					
Initial State: 
					
Input: 
					
Output: 
					
How test will be performed: 

\end{enumerate}

\paragraph{Capacity}

\begin{enumerate}

\item{SS-C-1\\}

Type: Non-Functional, Static, Manual
					
Initial State: 
					
Input/Condition: 
					
Output/Result: 
					
How test will be performed: 

\end{enumerate}

\subsubsection{Maintainability and Support}

\paragraph{Source Code}

\begin{enumerate}

\item{SS-SC-1\\}

Type: Non-Functional, Static, Manual
					
Initial State: 
					
Input/Condition: 
					
Output/Result: 
					
How test will be performed: 
					
\item{SS-SC-2\\}

Type: Non-Functional, Static, Manual
					
Initial State: 
					
Input: 
					
Output: 
					
How test will be performed:

\item{SS-SC-3\\}

Type: Non-Functional, Static, Manual
					
Initial State: 
					
Input: 
					
Output: 
					
How test will be performed: 

\end{enumerate}

\subsubsection{Security}

\paragraph{File Integrity}

\begin{enumerate}

\item{SS-FI-1\\}

Type: Non-Functional, Static, Manual
					
Initial State: 
					
Input/Condition: 
					
Output/Result: 
					
How test will be performed: 
					
\item{SS-FI-2\\}

Type: Non-Functional, Static, Manual
					
Initial State: 
					
Input: 
					
Output: 
					
How test will be performed: 

\end{enumerate}

\paragraph{Access, Privacy, and Immunity}

\begin{enumerate}

\item{SS-API-1\\}

Type: Non-Functional, Static, Manual
					
Initial State: 
					
Input/Condition: 
					
Output/Result: 
					
How test will be performed: 
					
\item{SS-API-2\\}

Type: Non-Functional, Static, Manual
					
Initial State: 
					
Input: 
					
Output: 
					
How test will be performed: 

\item{SS-API-3\\}

Type: Non-Functional, Static, Manual
					
Initial State: 
					
Input: 
					
Output: 
					
How test will be performed:

\end{enumerate}

\subsection{Traceability Between Test Cases and Requirements}

\section{Tests for Proof of Concept}

The primary feature in the PoC is the application's ability to recognize a Sudoku image and display the recognized digits on a web-based front-end. Therefore, tests for the PoC will focus on image extraction, digit recognition, and front-end presentation.

\subsection{Board image extraction}
		
This portion of the testing focuses on the extraction of Sudoku board and cell image data from a set of 10 photographs of varying perspective, brightness, and clarity. The output is compared manually with the expected output in each case, since automatic testing of computer vision results is an advanced topic beyond the scope of this project.

\begin{enumerate}

\item{poc-ie1\\}

Type: Functional, Dynamic, Manual
					
Initial State: SudokuCV object is instantiated
					
Input: 10 different photographs of Sudoku boards, and 3 photographs without Sudoku
					
Output: flattened and cropped images of Sudoku boards, or raise error if no board is detected.
					
How test will be performed: SudokuCV object is supplied with input image through the recognize() method with the show\_image parameter set to true. This displays the input and output images for manual validation.

\item{poc-ie2\\}

Type: Functional, Dynamic, Manual
					
Initial State: SudokuCV object is instantiated
					
Input: 3 photographs without a fully visible Sudoku puzzle
					
Output: error string indicating that no Sudoku board can be extracted.
					
How test will be performed: Three images with an assortment of problems are supplied to test the error handling of the SudokuCV recognize() method. Specifically, one input image does not have a Sudoku board, another is too small to extract any features from, and a third cropped to include only a part of a board.
					
\item{poc-ie3\\}

Type: Functional, Dynamic, Manual
					
Initial State: SudokuCV object is instantiated
					
Input: 10 different photographs of Sudoku boards
					
Output: An array of 81 images each cropped from a cell on the 9x9 Sudoku board
					
How test will be performed: array of numpy output is displayed using the Opencv imshow() method and manually validated against the input image.

\end{enumerate}

\subsection{Digit recognition}
\begin{enumerate}

\item{poc-dr1\\}

Type: Non-functional, Dynamic, Automatic
					
Initial State: a neural network model is trained from the handwritten and printed digits data set
					
Input: 10000 MNIST handwritten numbers test set
					
Output: percentage accuracy of the model prediction
					
How test will be performed: cvtraining module compares the labels of the input data set against the prediction made by the trained model and returns the rate of correct predictions.

\item{poc-dr2\\}

Type: Non-functional, Dynamic, Manual
					
Initial State: SudokuCV object is instantiated with trained model
					
Input: 10 photographs of Sudoku boards with handwritten and printed numbers
					
Output: an integer array of size 81 with recognized digits
					
How test will be performed: SudokuCV object supplies the array of recognized digits (0 for empty cells) and a corresponding array of confidence rates for each index. Digits with a confidence greater than 75\% are displayed and manually compared against the input images to calculate recognition accuracy.

\end{enumerate}

\subsection{Front-end presentation}
\begin{enumerate}

\item{poc-fp1\\}

Type: functional, Dynamic, manual
					
Initial State: Flask app is running using an initialized SudokuCV object
					
Input: 10 photographs of Sudoku boards with handwritten and printed numbers
					
Output: HTML page displaying extracted board image and table of recognized digits 
					
How test will be performed: The photographs are uploaded one at a time through the web-based GUI, the output table is manually compared to the original photograph to validate that digits are displayed correctly in the Sudoku grid format.

\item{poc-fp2\\}

Type: functional, Dynamic, manual
					
Initial State: Flask app is running using an initialized SudokuCV object
					
Input: 3 invalid Sudoku photographs (too small, no Sudoku puzzle, and empty grid)
					
Output: HTML page displaying error and suggested solution
					
How test will be performed: The photographs are uploaded one at a time through the web-based GUI, the output is manually compared against the expected error for each input image.

\end{enumerate}

	
\section{Comparison to Existing Implementation}	
				
\section{Unit Testing Plan}
		
\subsection{Unit testing of internal functions}
The internal functions of the system mostly include the Sudoku solver algorithm itself, the Sudoku board generator, and the image recognition functions. To test such functions, it is simple enough to give these functions an input and ensure that the output is what is to be expected from that given input. Since most of the internal functions is written in Python and JavaScript, we will be utilizing the PyTest and Mocha automatic testing frameworks for each programming language respectively. Since the board generator creates random boards, it is impractical generate and test every possible board. Therefore, unit testing for Sudoku board generation will create a reasonably sized set of boards, between 100 to 200. Individual functions that check for solvability and uniqueness will be called upon these generated boards to confirm the reliability and validity of the generation system. 
		
\subsection{Unit testing of output files}		
Unit testing the output files of the system will be performed by inputting various Sudoku boards via images (or manual input) into the web application, and ensuring that the outputted solution follows the Sudoku rules and is complete. The outputted Sudoku boards can be scraped from the web application, transformed into an array format, and be used as an input into a function that tests whether a complete Sudoku board follows the Sudoku rules. 

\bibliographystyle{plainnat}

\bibliography{SRS}

\newpage

\section{Appendix}

This is where you can place additional information.

\subsection{Symbolic Parameters}

The definition of the test cases will call for SYMBOLIC\_CONSTANTS.
Their values are defined in this section for easy maintenance.

\subsection{Usability Survey Questions?}

This is a section that would be appropriate for some teams.

\end{document}
